\documentclass[a4paper,11pt]{book}
\usepackage[T1]{fontenc}																					% n�dvendig for fonter
\usepackage[latin1]{inputenc}																			% n�dvendig for ���
\usepackage{amsmath,amsfonts,amssymb,yfonts,mathrsfs,gensymb}			% n�dvendig for matematiske symboler
\usepackage{parskip}																							% blank linje mellom avsnitt istedenfor innrykk
\usepackage{enumerate}			% fancy enumerering
\usepackage{graphicx,textcomp,varioref}
\usepackage{cancel}					% kanseller ledd i math mode

\author{Ruben Spaans}
\title{My notes on algorithmic problem solving}
\date{\today}

\begin{document}

\maketitle

\tableofcontents

\chapter{Introduction}
\label{cha:intro}

In this book I will write down strategies I have used in problem solving so that I don't forget them.

Suggested outline:

\begin{itemize}
	\item Sorting algorithms (bubble sort, quicksort, mergesort, counting sort, radix sort and how to augment some of these algorithms to solve more advanced problems)
	\item Graph algorithms, including: single-pair shortest path, all-pairs shortest path, network flow (including mincost max-flow (and min-cut), bipartite matching, non-bipartite matching, ``tripartite matching'', assignment problem, feasible flow, circulation, multi-commodity flow), depth-first search, breadth-first search (and their variants), minimum spanning trees, discrete steiner trees, articulation point, bridge, strongly connected components, topological sort, eulerian path/cycle, hamilton path/cycle, planarity, longest path, informed search (A*). Mention common special cases about each problem kind and how to augment the algorithm to be more efficient in these cases. I guess it's best to split this into multiple chapters.
	\item Arithmetic (including biginteger, fast exponention, digit sum, sums of consecutive integers)
	\item Number theory
	\item (Computational) geometry (convex hull, 2D/3D area/volume/surface, intersections)
	\item Advanced arithmetic (matrices, linear algebra etc)
	\item Combinatorics and probability (permutations, partitions, sets, trees)
	\item Combinatorial game theory
	\item Greedy algorithms
	\item Dynamic programming
	\item Brute force (various methods to generate all possitilities, backtracking, branch and bound, pruning, meet-in-the-middle)
	\item String algorithms (search, palindromes, aho-corasick, suffix arrays/trees)
	\item Data structures (need to split this, there are many and they are different, such as disjoint-set, (binary) trees of all kinds, hashtables, buckets, min/max-heap and other heaps, binary decision diagram)
	\item Continuous optimization
	\item Common ad hoc themes (spiral, roman numerals, calendar/dates, poker)
	\item Simulation problems
	\item Parsing and grammars
	\item Boolean tricks
	\item Search techniques (not string search, but binary search, ternary search)
	\item Compression (huffman coding, RLE etc)
	\item Output-based problems
	\item 
\end{itemize}

\chapter{Arithmetic}
\label{cha:arithmetic}

This chapter contains discussion of problems that don't involve advanced mathematical concepts, only basic arithmetic.

\chapter{Discrete optimization, other}
\label{cha:discopt}

This chapter contains discussion of optimization problems that don't fit in any of the other chapters.

\section{Two-dimensional maximization}

\subsection{Example problem: UVa 12520 ``Square garden''}

{\bf Problem statement} Given a square grid consisting of $L\times L$ unit cells and a given integer $N$ with $0 \leq N \leq L^2$, colour $N$ cells such that the perimeter of the coloured cells is maximized. What is the length of the perimeter?

\begin{tabular}{l|l}
	\hline
		{\bf Input}, $L \text{ }N$ & {\bf output} \\
	\hline
		1 0 & 0 \\
		1 1 & 4 \\
		2 3 & 8 \\
		3 8 & 16 \\
	\hline
\end{tabular}

{\bf Discussion} One way to attack this problem is to break it down into several cases and solve them separately. An easy upper bound for the answer is $4N$ which happens when no two neighbouring cells are coloured. By colouring cells in a chessboard pattern we can achieve the upper bound for $N \leq \lfloor \frac{L^2+1}{2} \rfloor$.

For $N$ closer to $L^2$, let's look at the problem in reverse: Start with a completely coloured grid and uncolour cells until we have $N$ coloured cells. Ideally we want to uncolour cells that increase the perimeter by 4 each time. By creating a checkerboard pattern in the interior cells, we can uncolour up to $\lfloor \frac{(L-2)^2+1}{2} \rfloor$ cells until we run out of interior cells. Hence, If $N \geq L^2-\lfloor \frac{(L-2)^2+1}{2} \rfloor$ we can easily calculate the optimal perimeter.

After we have uncoloured all interior cells, we need to uncolour non-corner border cells. Each one of these will increase the perimeter by 2 (as long we don't create adjacent uncoloured cells). Here we have two cases for odd $L$: We can uncolour $\lfloor \frac{(L-2)^2+1}{2} \rfloor$ interior cells and have $4\lfloor \frac{L-2}{2} \rfloor$ non-corner border cells, or $\lfloor \frac{(L-2)^2}{2} \rfloor$ interior cells and $4\lfloor \frac{L-1}{2} \rfloor$ non-corner border cells. We can simply calculate both alternatives and take the maximum. If we have run out of non-corner border cells, we need to take the corners. Uncolouring corner cells won't increase the perimeter. For even $L$, there are two available corner cells, and for odd $L$ there are 4 corner cells in the first case and none in the second case.

This covers all cases, because if we have removed all possible interior cells and border cells, we have arrived at the chessboard pattern.

\end{document}
