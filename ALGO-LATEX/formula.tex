\documentclass[a4paper,11pt]{article}
\usepackage[T1]{fontenc}
\usepackage[latin1]{inputenc}
\usepackage{amsmath,amsfonts,amssymb,yfonts,mathrsfs,gensymb}
\usepackage{parskip}											% newline between paragraphs
\usepackage{enumerate}										%	fancy enumerate
\usepackage{graphicx,textcomp,varioref}
\usepackage{cancel}												% cancel terms in math mode
\usepackage{tipa}

\begin{document}

\section{Number theory}

pick's theorem

pythagorean triplets


\section{(Computational) Geometry}

A \emph{median} in a triangle is a line from a vertex to the midpoint of its opposing edge. These relationships exist between side lengths $a,b,c$ and medians $m_a,m_b,m_c$:

\subsection{Triangle}

Triangle area given sides

\subsubsection{Medians of a triangle}

The median of a triangle is a line segment going from a vertex to the midpoint of the opposing line.

\begin{align*}
	m_a &= \sqrt{\frac{2b^2+2c^2-a^2}{4}}\\
	m_b &= \sqrt{\frac{2a^2+2c^2-b^2}{4}}\\
	m_c &= \sqrt{\frac{2a^2+2b^2-c^2}{4}}\\
	a &= \frac{2}{3}\sqrt{-m_a^2 + 2m_b^2 + 2m_c^2} \\
	b &= \frac{2}{3}\sqrt{-m_b^2 + 2m_a^2 + 2m_c^2} \\
	c &= \frac{2}{3}\sqrt{-m_c^2 + 2m_b^2 + 2m_a^2}
\end{align*}

When going from medians to side lengths, a triangle is valid if all expressions under the roots are positive, and $a+b>c$ for some permutation of $a,b,c$.

\section{Combinatorics}

Number of ways to pick $k$ objects (ordered) from a set of $n$: $$n P r = \frac{n!}{(n-k)!}$$

Calculate the last non-zero digit of $\prod a_i$: Let $c=\sum (f(a_i)-g(a_i))$ where $f(x)$ is the largest power of 2 that divides $x$ and $g(x)$ is the largest power of 5 that divides $x$. Cast out all 2's and 5's from $a_i$, calculate the product and keep the last digit. Then, if $c<0$, the last digit is 5, and if $c>0$, multiply the last digit with $2^c$.

\end{document}
