\documentclass[a4paper,11pt]{article}
\usepackage[T1]{fontenc}																					% n�dvendig for fonter
\usepackage[latin1]{inputenc}																			% n�dvendig for ���
\usepackage{amsmath,amsfonts,amssymb,yfonts,mathrsfs,gensymb}			% n�dvendig for matematiske symboler
\usepackage{parskip}																							% blank linje mellom avsnitt istedenfor innrykk
\usepackage{enumerate}			% fancy enumerering
\usepackage{graphicx,textcomp,varioref}
\usepackage{cancel}					% kanseller ledd i math mode

\author{Ruben Spaans}
\title{Algorithm}
\date{\today}

\begin{document}

\section{Help, I got ``Wrong answer!''}

\subsection{Understanding}

\begin{itemize}
	\item If you have mysterious errors, read the problem statement again. Make sure you're solving the correct problem.
\end{itemize}

\subsection{Choice of algorithm}

\begin{itemize}
	\item Keep things simple! Choose the simplest algorithm that will run in time.
	\item If a lot of teams have solved a problem, there is likely to be an easier algorithm.
\end{itemize}

\subsection{Correctness of algorithm}

\begin{itemize}
	\item Your fast greedy algorithm which works for the examples need not be correct. Prove it, or don't use it! In general, it is dangerous to use an algorithm which doesn't test all possibilities.
	\item Check if your cute and super-short code optimization is correct. It is insidiously hard to get these correct. Don't attempt to do this unless you are certain that the algorithm is correct and there is no other way to avoid ``Time limit exceeded.''
	\item Always be critical to your choice of algorithm. Try as hard as you can to find input which breaks it.
\end{itemize}

\subsection{Corner cases, ``gotcha'' elements}

\begin{itemize}
	\item Are you handling all corner cases? Try the minimum/maximum inputs, graph with no nodes, graph with one node, graph with no edges, empty set, one-element set, and other kinds of cases.
	\item Don't obey restrictions not mentioned in the problem statement. Move out of the grid unless prohibited!
	\item Check if parity or other properties make the problem much simpler.
	\item Beware the old ``input trap''. If you are given $a$ and $b$ and are asked to do something with this range, be sure to ensure that $a\leq b$ by swapping.
\end{itemize}

\subsection{Arrays}

\begin{itemize}
	\item Let arrays have the size you need plus 10 or so, and also precalculate stuff up past the original maximal constraints. This headroom can save you if you have off-by-one errors on the max size.
\end{itemize}


\subsection{Graph theory}

\begin{itemize}
	\item In an adjacency matrix, remember to set the diagonal to 0 to represent that it costs nothing to not move.
	\item In min-cost maxflow, remember to set \texttt{cost[j][i]=-cost[i][j]}!
	\item In BFS, remember to set the start node as visited with distance 0.
	\item In BFS, return a distance of 0 if the start node is a goal node!
	\item In Floyd-Warshall, use indexes $k,i,j$ in this order when checking if it is faster to go from $i$ to $j$ via $k$.
	\item Beware of multiple edges between the same pair of nodes!
\end{itemize}

\subsection{Number theory}

\begin{itemize}
	\item Remember that \texttt{a \% b} for $\texttt{a}<0$ is nonpositive! Use \texttt{(a \% b + b) \% b}, for instance, or bitwise AND.
\end{itemize}

\subsection{Dynamic programming}

\begin{itemize}
	\item Don't use DP if the recurrence has cycles! Consider doing a simultaneous DP/gaussian elimination where the gaussian elimination operates within each strongly connected component, working its way through the topologically sorted component graph.
\end{itemize}

\subsection{Debugging}

\begin{itemize}
	\item Be careful with indexes! Check that \texttt{i} and \texttt{j} aren't switched. Make sure that the \texttt{cx} in \texttt{x2~=~cx~+~dx[d]} isn't forgotten.
	\item If the end result is a \texttt{long long}, be sure to let the intermediate variables be \texttt{long long} as well. \texttt{int} multiplication overflow is an extremely common mistake.
	\item Be careful when mapping 2D coordinates to a flat 1D-array! Given a 2D system of size $x \times y$, the correct mapping of $(i,j)$ is $i+j\cdot x$. Take care to not use $y$. You might even want to use a square array and hardcode the dimensions.
	\item Please select easy-to-implement solutions with few/no special cases to consider (as long as they will be fast enough to pass).
	\item Getting different results on the same test case? Is the answer to the first test case correct, while the rest is nonsense? Remember to clean up variables between test cases!
\end{itemize}

\section{Help! I got ``Time limit exceeded!'' }

\subsection{Algorithmic complexity}

\begin{itemize}
	\item Check the worst case (number of test cases multiplied with the maximum constraints). If you are doing more than $10^9$ operations, reconsider your algorithm.
	\item If the problem is of a combinatorial nature, try to think of a test case which invokes worst-case behaviour.
	\item Try to reduce the polynomial exponent by preprocessing, for instance DP for prefix sums.
\end{itemize}


Look at the constraints and see if you can get any hints from them. Look at the following table to get ideas on which algorithm to choose if $n$ is the main parameter.

\begin{tabular}{|c|c|l|}
\hline
$n$ & Time complexity & Method \\
\hline
11 & $n!$ & Try all permutations \\
\hline
16 & $3^n$ & Try all ways of setting 3 states \\
 & & Try all subsets of subsets \\
\hline
22 & $2^n$ & Meet in the middle (snoob) \\
 & & Bitmask DP \\
 & & Try all subsets \\
 & & Backtracking with two choices \\
\hline
40 & $2^{n/2}$ & Meet in the middle: split problem in half, \\
 & & solve each half separately and recombine \\
\hline
? TODO & $1.5^n$ & Try all integer partitions \\
\hline
500 & $n^3$ & Matrix stuff \\
\hline
8000 & $n^2$ & \\
\hline
$10^5$ & $n^{1.5}$ & Square root bins \\
\hline
$10^6$ & $n \log n$ & Sorting \\
 & & Binary tree \\
\hline
$5\cdot 10^7$ & $n$ & Try all $n$ \\
 & & Linear scan \\
\hline
$10^{18}$ & $\log n$ & Exponentiation \\
 & & Discrete binary search \\
\hline
any $n$ & $O(1)$ & Clever stuff \\
\hline
\end{tabular}

If your algorithm has higher constraints than the suggested $n$ in the left column, try to come up with a new idea (or prune!). The above $n$ assumes very efficient implementations in C. Lower somewhat for other languages.

\subsection{Calculating stuff twice}

\begin{itemize}
	\item If something non-trivial is calculated more than once, calculate it once and store the results! For instance, it the problem is to solve several instances of the 8-puzzle, search once from the solved state, and look up each instance directly.
\end{itemize}

\subsection{Dynamic programming}

\begin{itemize}
	\item When doing memoization, make sure to distinguish between ``not yet calculated'' and no solution. Do not assign -1 to both! This will cause all subproblems with no solution to not be cached!
\end{itemize}

\subsection{Game theory}

\begin{itemize}
	\item Don't calculate grundy numbers if just \texttt{xor}-ing nim values is enough (and the nim-values are easily obtainable without doing the mex calculations). This happens in particular if a game cannot be split into multiple subgames.
\end{itemize}

\subsection{Optimization}

\begin{itemize}
	\item Don't put loops in the wrong order. When traversing a multidimensional array, let the innermost loop run through the rightmost dimension. Doing it the wrong way hurts cache performance and it could potentially be the only thing preventing correct answer in time.
\end{itemize}

\section{Help! I'm stuck! I can't solve the problem etc}

Here are some algorithms to try.

\begin{itemize}
	\item Brute force or backtracking
	\item Dynamic programming
	\item Maximum flow
	\item Breadth-first, depth-first or priority-first search
	\item Shortest path
	\item Gaussian elimination
	\item Binary search
	\item Sweep line/interval compression
	\item Meet-in-the-middle
	\item Binary tree data structure thingy
	\item etc.
\end{itemize}

Try to formulate the problem in terms of:
\begin{itemize}
	\item Graph theory
	\item System of equations
	\item State space traversal (try to obtain a DAG and run DP)
	\item Linear programming
	\item etc.
\end{itemize}

Try to combine algorithms by assuming that some variables are known, and try to find ``inner'' and ``outer'' algorithms for each ``half''. Typical ``outer'' algorithms are binary search, sorting and trying all combinations of some kind.

\section{Theorems and stuff}

\subsection{General tips}

\begin{itemize}
	\item When manipulating math formulas, try many things. If a function has many variables, try all ways of fixing a subset of variables and observe the relationship in return value when varying the free variables.
	\item When solving a problem involving bouncing and reflection, imagine that the object is instead continuing in a straight line in a mirrored environment. This way of viewing the problem domain can lead to new insights and an easier solution.
\end{itemize}

\subsection{Number theory}

\begin{itemize}
	\item Think a little bit before jumping into coding. Simple observations can make the code much simpler, shorter, faster and easier to implement.
\end{itemize}

\subsection{Probability}

\begin{itemize}
	\item Warning, don't underestimate the difficulty when conditional probabilities are involved. Be sure of what you're doing and stay sharp!
	\item For DP problems of the type ``what's the probability of escaping through some ordeal'', consider not normalizing the probabilities, and do subtraction of non-normalized probabilities.
	\item Given a random permutation of length n, the probability of it being in a cycle of length $k$ is $\frac{1}{n}$ for any $1\leq k \leq n$. TODO this doesn't make sense, figure out what I really meant and rewrite it.
\end{itemize}

\subsection{Graph theory}

\begin{itemize}
	\item If the graph is undirected, don't even think about finding strongly connected components.
	\item Max-clique on $G$ is equivalent to max independent set on the complement of $G$.
\end{itemize}

\end{document}
